%/* vim: set filetype=latex */

% Demo project. Uses Komascript >=3.0. 
% which is not included in texlive 2008
% (1) copy dist to texlive/texmf-local 
% (2) texhash
%\documentclass[BCOR=8.5mm,DIV=calc,open=right,pagesize=auto,a5paper]{scrbook}
%\documentclass[12pt,BCOR=8.5mm]{scrbook}
\documentclass[11pt,BCOR=8.5mm]{scrartcl}

% Use utf-8 encoding for foreign characters
\usepackage[utf8]{inputenc}
\title{Amateurfunkprüfung}
\subtitle{Prüfungsteil "`Betriebstechnik und Vorschriften"'}
\author{Mathias Dalheimer \texttt{<md@gonium.net>}}
%% PDF SETUP
\usepackage[pdftex, bookmarks, colorlinks, breaklinks,
pdftitle=\title,pdfauthor={Mathias Dalheimer},plainpages=false]{hyperref}  
\hypersetup{linkcolor=blue,citecolor=blue,filecolor=black,urlcolor=blue,plainpages=false} 

\clubpenalty=3000 % adjust for widows and orphans 10000 is max
\widowpenalty=3000 % adjust for widows and orphans 10000 is max
%% Font Adjustments
\usepackage[LY1]{fontenc}
\usepackage[T1]{fontenc}
%\usepackage{adobegaramond}
%\usepackage{gillsans}
%\renewcommand{\rmdefault}{HoeflerText}

% Use URW Garamond No. 8 as a default font. (getnonfreefonts)
%\usepackage[urw-garamond]{mathdesign}
%\renewcommand{\rmdefault}{ugm}
% Optima as a sans serif font.
%\renewcommand*\sfdefault{uop}

\newcommand*\imgwidth{0.8\textwidth}

\usepackage{ifthen}
\newboolean{InternalVersion}
\setboolean{InternalVersion}{true}
%\setboolean{InternalVersion}{false}

\usepackage{listings}  
\lstset{numbers=left, numberstyle=\tiny,numbersep=5pt}  
\lstset{language=Xml, basicstyle=\small, frame=shadowbox}  

\usepackage{algorithmic}
\usepackage{algorithm}
%\numberwithin{algorithm}{chapter} 
%\newcommand{\theHalgorithm}{\arabic{algorithm}}

\usepackage[protrusion=true,expansion=true]{microtype}

%% Page Header
\usepackage{scrpage2}

% Recalculate page setup based on new font.
%\KOMAoptions{DIV=last}
%\KOMAoptions{draft=true}

% Versals. 
%\usepackage{lettrine}
% Misc packages
\usepackage{url}
\usepackage{graphicx}
\usepackage{todonotes}
%\usepackage[english]{babel}
\usepackage{ngerman}
\usepackage{blindtext}
\usepackage{subfigure}

\hyphenation{scho-oner Sur-vivor ap-pen-dix
Strom-ver-brauchs-informationen}

\usepackage{marvosym} % bei der Schrift enthalten
\newcommand*\euro{\textup{\EUR}}


%% Chapterstyle
%\renewcommand*{\chapterformat}{---\hskip.5cm\thechapter\hskip.5cm---}
%\KOMAoption{headings}{small,twolinechapter}
%\setkomafont{chapter}{\Large\sffamily\centering}
%\setkomafont{dictum}{\normalfont}
%\renewcommand*{\dictumwidth}{.8\textwidth}

%% Main.
\begin{document}
\maketitle
%\clearscrheadfoot
%\automark[chapter]{section}
%\lehead[]{\pagemark\hskip.5cm\vrule\hskip.5cm\title}
%\rohead[]{\headmark\hskip.5cm\vrule\hskip.5cm\pagemark} 
\title
%\pagestyle{scrplain}
%\begin{titlepage}
%\input{frontpage}
%\end{titlepage}
%% frontmatter
%%\input{frontmatter}
%%\parindent.0cm
%%\parskip.4cm
%%\input{pagetwo}
%\parindent.4cm
%\parskip.0cm
%\pagestyle{scrheadings}
%
\section{Internationales Buchstabieralphabet}

\begin{table}[h]
  \centering
\begin{tabular}[h]{|l|l|}
  \hline
  Buchstabe & Schlüsselwort \\
  \hline
  \hline
  A & Alpha \\
  B & Bravo \\
  C & Charlie \\
  D & Delta \\
  E & Echo \\
  F & Foxtrott \\
  G & Golf \\
  H & Hotel \\
  I & India \\
  J & Juliett \\
  K & Kilo \\
  L & Lima \\
  M & Mike \\
  N & November \\
  O & Oscar \\
  P & Papa \\
  Q & Quebec \\
  R & Romeo \\
  S & Sierra \\
  T & Tango \\
  U & Uniform \\
  V & Victor \\
  W & Whiskey \\
  X & X-Ray \\
  Y & Yankee \\
  Z & Zulu \\
  \hline
\end{tabular}
\end{table}

\section{Der Q-Schlüssel}

Alle Zeiten in UTC! Nur im Telegrafiefunkverkehr verwenden! Skala 1-5: 1
entspricht wenig, 5 entspricht viel.

\begin{table}[h]
  \centering
  \begin{tabular}{| l | p{4.3cm} | p{4.3cm} | l |}
  \hline
  Q-Code & ! & ? & Merke \\
  \hline
  \hline
  QRK & Die Verständlichkeit ihrer Zeichen ist (1-5) & Wie ist die
  Verständlichkeit meiner Zeichen? & Verständlichkeit \\
  \hline
  QRM & Ich werde gestört (1-5) & Werden Sie gestört? & Matsch \\
  \hline
  QRN & Ich werde durch atmosphärische Störungen beeinträchtigt (1-5) &
  Werden sie durch atmosphärische Störungen beeinträchtigt? &
  Noise \\
  \hline
  QRO & Erhöhen Sie die Sendeleistung & Soll ich die
  Sendeleistung erhöhen? & Output \\
  \hline
  QRP & Verringern Sie die Sendeleistung & Soll ich die
  Sendeleistung vermindern? & Pipi \\
  \hline
  QRT & Stellen Sie die Übermittlung ein. & Soll ich die
  Übermittlung einstellen? & Terminate \\
  \hline
  QRV & Ich bin bereit & Sind Sie bereit? & Bin bereit \\
  \hline
  QRX & Ich werde Sie um \ldots Uhr wieder rufen. & Wann werden
  Sie mich wieder rufen? & Pause \\
  \hline
  QRZ & Sie werden von \ldots gerufen & Von wem werde ich
  gerufen? & Wer ruft? \\
  \hline
  QSB & Die Stärke Ihrer Zeichen schwankt. & Schwankt die Stärke
  meiner Zeichen? & Bold \\
  \hline
  QSL & Ich gebe Ihnen Empfangsbestätigung. & Können Sie mir
  Empfangsbestätigung geben? & \\
  \hline
  QSO & Ich kann mit \ldots unmittelbar verkehren. & Können Sie
  mit \ldots verkehren? & \\
  \hline
  QSY & Gehen Sie auf eine andere Frequenz über & Soll ich auf
  eine andere Frequenz übergehen? & \\
  \hline
  QTH & Mein Standort ist \ldots Breite, \ldots Länge & Welches
  ist Ihr Standort? & Home \\
  \hline
\end{tabular}
\end{table}

\end{document}
